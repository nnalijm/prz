\documentclass[a4paper, 12pt]{book}


\usepackage{polski}
\usepackage{amsmath}
\usepackage{graphicx}
\usepackage{blindtext}
\usepackage{caption}
\usepackage{lipsum}
\usepackage{hyperref}

\title{Narzedzia dla programistów ćw. 4\\Sprawozdanie\\Tworzenie dokumentów \LaTeX}
\author{Grupa №11}
\date{}


\begin{document}
\maketitle
\tableofcontents
\listoffigures
\listoftables

\part{To jest tytuł I części}
\chapter{To jest tytuł rozdziału 1}


\textbf{\blindtext \\ \\}
\ldots

\blindtext \\ \\
\ldots
\clearpage

\section{To jest tytuł sekcji 1}
\blindtext
\subsection{To jest tutył podsekcji 1}
\blindtext
\subsection{To jest tytuł podsekcji 2}
\blindtext
\clearpage
\section{Krój pisma} \label{pismo}
Teraz będą przykłady użycia różnych krojów pisma. Odpowiedź do pkt. 4.2.

\textbf{To jest przykładowe zdanie napisane czcionką pogrubioną.}

\emph{To jest przykładowe zdanie napisane czcionką pochyloną.}

\textit{To jest przykładowe zdanie napisane kursywą.}

\underline{To jest przykładowe zdanie napisane z podkreśleniem.}

\verb|To jest przykładowe zdanie napisane jako maszynopis.|

Teraz będzie {\Large większy rozmiar czcionki.} Lub inaczej.

Teraz będzie {\Large większy rozmiar czcionki.} Zobacz różnicę.

Teraz będzie {\small mniejszy rozmiar czcionki.} Lub inaczej.

Teraz będzie {\small mniejszy rozmiar czcionki.} Zobacz różnicę.

\subsection{Środowiska}
To będzie przykładowe wyliczenie:
\begin{enumerate}
\item to po pierwsze, 
\item to po drugie,
\item po trzecie i już koniec
\end{enumerate}
To będzie przykładowe wypunktowanie:
\begin{itemize}
\item to po pierwsze,
\item to po drugie,
\item po trzecie i już koniec.
\end{itemize}
A teraz wstawiamy rysunek.

\begin{center}
\includegraphics[width=95mm, height=125mm]{zakaz.jpg}
\captionof{figure}{\textit{,,Kultowy rysunek naszego Wydziału''}}
\label{rys:da}
\end{center}
\clearpage
A teraz wstawimy tabelę
\begin{table}[h]
\caption{\textit{Przykładowa tabela}}
\label{tab:da}
\begin{center}
\begin{tabular}{| c c | c |}
\hline
t1 & t0 & data type\\
\hline
0 & 0 & \emph{bool}(\verb|B|)\\
0 & 1 & -- \\
1 & 0 & \emph{integer}(\verb|I|) \\
1 & 1 & \emph{long}(\verb|L|)\\
\hline
\end{tabular}
\end{center}
\end{table}

\subsection{Wzory matematyczne}
Teraz napiszemy wzór na pole koła
\begin{equation}
S = \pi r^{2}
\end{equation}
A teraz napiszemy wzór na wyróżnik $\Delta$ trójmianu kwadratowego
\begin{equation}
\label{formula}
\Delta = b^{2} - 4 \cdot a \cdot c
\end{equation}
oraz jego miejsca zerowe w postaci ułamka zwykłego.
\begin{equation}
x_1 = \frac{-b-\sqrt{\Delta}}{2\cdot a}
\end{equation}

\begin{equation}
x_2 = \frac{-b+\sqrt{\Delta}}{2\cdot a}
\end{equation}
\clearpage

\chapter{To jest tytuł rozdziału 2}
\section{Odwołania}
Na rysunku \hyperref[rys:da]{1.1}, który pojawił się w rozdziale \hyperref[pismo]{1.2}, na stronie 14 pojawił się obrazek, jaki można zobaczyć na drzwiach dr inż. Dariusza Klepackiego, z którym spotkacie się w przyszłym semestrze. Jak nauczycie się matematyki w minimalnym stopniu, choćby wzór \hyperref[formula]{1.1}, który pojawił się w rozdziale \hyperref[pismo]{1.2} na stronie 13. Do tabeli \hyperref[tab:da]{1.1} z rozdziału \hyperref[pismo]{1.2} ze strony 15 już się nie odwołamy, bo czas zajęć dobiega końca.\footnote{Przypis dolny}
\subsection{Znaki specjalne}
Właśnie zakodowałem znaki specjalne:

\textdollar, \#, \%, \^{}, \&, ", \verb|[|, \verb|]|, \textbackslash, \~{}, \{, \}, \vert.
\subsection{Spisy}
W miejsce, gdzie chcemy wstawić spisy treści, rysunków, tabel, należy wpisać odpowiednio trzy polecenia.\\
\\
\verb|\tableofcontents|\\
\verb|\listoffigures|\\
\verb|\listoftables|
\clearpage
\section{Sekcja 2}
\subsection{Podsekcja 1}
\subsection{Podsekcja 2}
\part{To jest tytuł II części}
\chapter{To jest tytuł rozdiału 3}
\section{To jest tytuł sekcji 1}
\subsection{To jest tytuł podsekcji 1}
\subsection{To jest tytuł podsekcji 2}
\section{Sekcja 2}
\subsection{Podsekcja 1}
\subsection{Podsekcja 2}
\chapter{To jest tytuł rozdziału 4}
\section{Sekcja 1}
\subsection{Podsekcja 1}
\subsection{Podsekcja 2}
\section{Sekcja 2}
\subsection{Podsekcja 1}
\subsection{Podsekcja 2}
\part{To jest tytuł III części}
\chapter{To jest tytuł rozdziału 5}
\section{To jest tytuł sekcji 1}
\subsection{To jest tytuł podsekcji 1}
\subsection{To jest tytuł podsekcji 2}
\section{Sekcja 2}
\subsection{Podsekcja 1}
\subsection{Podsekcja 2}
\chapter{To jest tytuł rozdziału 6}
\section{Sekcja 1}
\subsection{Podsekcja 1}
\subsection{Podsekcja 2}
\section{Sekcja 2}
\subsection{Podsekcja 1}
\subsection{Podsekcja 2}

\end{document}
