\documentclass[10pt]{article}

\usepackage[left=20mm, right=20mm, top=20mm, bottom=20mm]{geometry} \usepackage{polski}
\usepackage{graphicx}
\usepackage{xcolor}
\usepackage{hyperref}
\usepackage{caption}
\usepackage{amsmath}
\usepackage{amssymb}
\usepackage{enumitem}
\usepackage{mathtools}


\title{\textbf{Narzedzia dla programistów ćw. 4\\
Tworzenie dokumentów \LaTeX}}
\author{Grupa №11}
\date{}

\begin{document}
\maketitle
\thispagestyle{empty}
\clearpage
\tableofcontents
\listoffigures
\listoftables
\clearpage
\thispagestyle{empty}

\section*{{\huge Ćwiczenie 4}}
\section*{{\huge Tworzenie dokumentów \LaTeX}}
\section{Rozpoczęcie pracy}
Ćwiczenie napisane za pomocą edytora \emph{VIM} oraz tex interpretera \emph{pdflatex}.
\section{Krój pisma}
\begin{enumerate}
\item Napisałem fragment czcionką \textbf{pogrubioną}.
\item Napisałem fragment czcionką \emph{pochylona}.
\item Napisałem fragment tekstu \textit{kursywą}.
\item Napisałem fragment tekstu \underline{z podkreśleniem}.
\item Za pomocą polecenia \verb|\verb!| okresliłem fragment tekstu zamieszczony jako ,,\verb|maszynopis|''.
\item Użyłem większego rozmiaru czcionku dla {\huge wybranego fragmentu tekstu}.
\item Użyłem mniejszego rozmiaru czcionki dla {\footnotesize wybranego fragmentu tekstu}.
\end{enumerate}
\section{Środowiska}
\begin{enumerate}[resume]
\item Zdefiniowałem wyliczenie
\begin{enumerate}
\item punkt 1
\item punkt 2
\item punkt 3
\end{enumerate}
\item Określiłem wypunktowanie
\begin{itemize}
\item punkt 1
\item punkt 2
\item punkt 3
\end{itemize}
\item Zdefiniowałem środowisko rysunku wyrównanego do środka strony oraz podpisu pod nim.
\begin{center}
\begin{equation}
\includegraphics[width=19mm, height=19mm]{tex.png}
\label{ris:da}
\end{equation}
\captionof{figure}{\textit{{\TeX} logo}}
\end{center}
\clearpage
\item Utworzyłem środowisko zawierające tabelę o 3 kołumnach i 5 wierszach. Dodaj tytuł tabeli(nad nią). Określiłem etykietę do tabeli.
\begin{table}[h]
\caption{\textit{Przykładowy tytuł tabeli}}
\label{tab:da}
\begin{center}
\begin{equation}
\begin{tabular}{|c|c|c|}
    \hline
    A & B & C \\
    \hline
    D & E & F \\
    \hline
    G & H & I \\
    \hline
    J & K & L \\
    \hline
    M & N & O \\
    \hline
\end{tabular}
\end{equation}
\end{center}
\end{table}
\end{enumerate}
\section{Wzory matematyczne}
\begin{enumerate}[resume]
\item Napisałem wzór na pole koła. Zdefiniowałem mu etykieta:
\begin{center}
\begin{equation}
S = \pi r^{2}
\end{equation}
\end{center}
\item Napisałem wzór na wyróżnik $\Delta$ trójmianu kwadratowego oraz miejsca zerowe w postaci ułamka zwykłego.
\begin{center}
\begin{equation}
\label{formula}
\Delta = b^{2} - 4\cdot a\cdot c
\end{equation}
\begin{equation}
x_1 = \frac{-b -\sqrt{\Delta}}{2\cdot a}
\end{equation}
\begin{equation}
x_2 = \frac{-b +\sqrt{\Delta}}{2\cdot a}
\end{equation}
\end{center}
\end{enumerate}
\section{Odwołania}
\begin{enumerate}[resume]
\item Napisałem zdanie wykorzystujące odwołania do etykiet zdefiniowanych w punktach \hyperref[ris:da]{10}, \hyperref[tab:da]{11}, \hyperref[formula]{12}.
\item Utworzyłem przpis dolny w dokumencie pojawiający się na końcu strony.
\footnote{Przypis dolny}
\end{enumerate}
\section{Znaki specjalne}
\begin{enumerate}[resume]
\item Zakodowałem w tekście występowanie następujących znaków: \textdollar, \#, \%, \^{}, \&, ", [\, ]\ , $ \backslash $, \~{}, $\lbrace, \rbrace, \mid, \textunderscore, @$.
\end{enumerate}
\section{Spisy}
\begin{enumerate}[resume]
\item Dodałem spis treści dokumentu, spis rysunków i spis tabel.
\end{enumerate}
\end{document}
